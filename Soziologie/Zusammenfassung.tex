\documentclass{scrartcl}

\usepackage[utf8]{inputenc}
\usepackage[ngerman]{babel}
\usepackage{hyperref}
\usepackage{amssymb}
\usepackage{mathtools}
\usepackage{gensymb}
\usepackage{float}
\usepackage{tikz}
\usetikzlibrary{positioning}
\usetikzlibrary{arrows}

\usepackage[a4paper, left=1cm, right=1cm, top=2cm, bottom=3cm]{geometry}

\DeclarePairedDelimiter\abs{\lvert}{\rvert}%
\DeclarePairedDelimiter\norm{\lVert}{\rVert}%
\DeclarePairedDelimiter\inner{\langle}{\rangle}%
\newcommand{\ffrac}[2]{\ensuremath{\frac{\displaystyle #1}{\displaystyle #2}}}

\begin{document}
\section*{Vorlesung 2: Einführung}
\subsection*{Was ist Soziologie?}
\begin{itemize}
    \item
        Gegenstand der Soziologie: Die ärgerliche Tatsache der Gesellschaft erzwingt und ermöglicht gemeinsames Handeln durch Gewebe aus Bedeutung, Erwartung und Verhalten in wechselseitiger Orientierung (Berger 2011, 47)
    \item
        Wie kommt ein Bedeutungsgewebe zustande, das gemeinsames (wechselseitiges aufeinander bezogenes) Handeln ermöglicht und erzwingt?\\
        $\rightarrow$ Durch die kulturelle Festlegung von Erwartungen an eine soziale Position (z.B. Ärztin, Lehrerin)
    \item
        Soziale Positionen vermitteln keine individuellen Kentnisse der Person. Sie sind auch nicht individuell, sondern sozial (überindividuell)
    \item
        Typische Verhaltensweise einer Position = soziale Rolle
    \item
        Indem Individuen soziale Rollen spielen, entsteht \textbf{Verlässlichkeit, Dauerhaftigkeit und Erwartbarkeit} im sozialen Verkehr
    \item
        Positionen und Rollen sind den Individuen äußerlich (sie sind soziale Tatsachen)
\end{itemize}

\begin{itemize}
    \item
        Wo ist das Soziale zu verorten?
        \begin{enumerate}
            \item
                Im Handeln der Individuen ($\rightarrow$ Wisschenschaft des sozialen Handelns)
            \item
                In der Wechselwirkung zwischen Individuen ($\rightarrow$ Wissenschaft von den sozialen Wechselwirkungen)
            \item
                Das Soziale tritt den Individuen als äußerer Zwang gegenüber ($\rightarrow$ Wissenschaft sozialer Institutionen)
        \end{enumerate}
        $\rightarrow$ Alle drei Antworten: es geht um soziale Beziehungen zwischen Menschen
\end{itemize}
\subsection*{Soziologisches Bewusstsein}
\begin{itemize}
    \item
        Misstrauen gegenüber offiziellen Problemdefinitionen
    \item
        Bewusstsein für die Pluralität der sozialen Wirklichkeit
        \begin{itemize}
            \item
                Abweichendes Verhalten kann aus der Perspektive einer anderen sozialen Gruppe Teil einer sinnvollen, respektablen kulturellen Praxis sein (z.B. Universitätsleben)
        \end{itemize}
    \item
        Relativieung des Geltungsanspruches kultureller Normen und Werte
        \begin{itemize}
            \item
                Die Entstehungs- und Geltungsbedingungen spezifischer Werte sind sinnvoller Gegenstand soziologischer Analysen, nicht aber ihre Wahrheit oder Richtigkeit
        \end{itemize}
\end{itemize}
$\rightarrow$ Gesellschaftlche Konstruktion der Wirklichkeit

\newpage

\section*{Vorlesung 3: Verhalten, Handeln, soziales Handeln, Normen}
\begin{itemize}
    \item
        soziale Institution: In der Soziologie versteht man unter Institutionen soziale Einrichtungen, die das Handeln der Menschen grundlegend bestimmen. Sie schränken Willkür und Beliebigkeit ein und geben den Menschen Verhaltenssicherheit (Handlungskompetenz). Institutionen in diesem Sinne sind zum Beispiel Familie, die Marktwirtschaft, das schulische Bildungswesen, das Gerichtswesen (Gerichtbarkeit) oder auch die Sprache.
\end{itemize}
\subsection*{Verhalten, Handeln, soziales Handeln}
\begin{itemize}
    \item
        Soziales Handeln: Handeln, das für den Handelnden subjektiv insofern sozial ist, als es sich auf das Verhalten anderer bezieht bzw. daran orientiert ist
    \item
        Handeln: Handeln ist eine Spezialform von Verhalten, nämlich subjektiv sinnhaftes Verhalten
    \item
        Verhalten: Jedes körperliche Geschehen in Raum und Zeit
    \item
        $\text{Soziales Handeln} \subset \text{Handeln} \subset \text{Verhalten}$
\end{itemize}
\subsection*{Handeln}
\begin{itemize}
    \item
        Handeln ist Verhalten mit subjektiv gemeintem Sinn (mit Motiven ausgestattet)
    \item
        Erfolgt in einer Situation (ist gerahmt)
    \item
        Es verfolgt ein Ziel und setzt zur Zielverfolgung intentionelle Mittel ein
    \item
        Hat intendierte und nicht intendierte Folgen
    \item
        Handeln wählt zwischen Alternativen
    \item
        Der Sinn einer Handlung wird von anderen gedeutet (interpretiert)
\end{itemize}
\subsection*{Soziales Handeln}
\begin{itemize}
    \item
        Soziales Handeln ist der Intention nach auf das Verhalten anderer gerichtet, dadurch in seinem Ablauf mitbestimmt und aud dieser Intention und Bestimmung des Handelns heraus versändlich erklärbar
    \item
        Die Sinnorientierungen des Handelns werden gesellschaftlich mitgeprägt (Wieso begrüßt man sich? $/rightarrow$ macht man hald so)
    \item
        Gesellschaftlicher Wandel ist über den Wandel der Sinnorientierung des Handelns erfassbar
\end{itemize}
\subsection*{Sinnorientierung sozialen Handelns}
\begin{itemize}
    \item
        Traditional: Sitten und Gebräuche (z.B. Heirat aus Tradition)
    \item
        Affektuell: Emotionen (z.B. Liebesheirat)
    \item
        Wertrational: Bewusster Glaube an den ethischen/religiösen etc. Eigenwert eines Sichtverhaltens, unabhängig vom Erfolg (z.B. Heirat weil persönliches erstrebenswertes Ideal)
    \item
        Zweckrational: Ausschließlich orientiert an als adäquat vorgestellten Mitteln für eindeutig erfasste Zwecke (z.B. Heirat um Steuern zu sparen)
    \item
        \textbf{Ideelle Typen, empirisch treten die Typen nicht rein auf}
\end{itemize}

\subsection*{Gesellschaft als sinnhafter Handlungszusammenhang}
\begin{itemize}
    \item
        Gesellschaft basiert auf individuellem Verhalten
    \item
        Verhalten weist Spielräume auf
    \item
        Konkretes Verhalten ist Folge einer Entscheidung
    \item
        Verhaltensselektion ist sozial mitbestimmt (nur sozial akzeptiertes Verhalten wird gezeigt)
\end{itemize}

\subsection*{Kritik an der Weberschen Handlungstypologie}
\begin{itemize}
    \item
        Check ich nicht
\end{itemize}

\subsection*{Pattern Variables (Talcott Parsons)}
\begin{itemize}
    \item
        Instrument für die strukturfunktionalistische Analyse einer Handlung
    \item
        Fünf dichotome Entscheidungsalternativen, zwischen denen ein Individuum bei der Handlung bewusst oder unbewusst wählen muss
        \begin{enumerate}
            \item
                Affektivität versus affektive Neutralität: Ist die Handlung von Gefühlen abhängig, oder ist es eine Handlung, die weitestgehend frei von Emotionen ist? Oft wird diese Variable auch als Wahl zwischen unmittelbarer Bedürfnisbefriedigung oder der Befriedigung eines langfristigen Bedürfnisses gesehen.
            \item
                Universalismus versus Partikularismus: Eine Handlung kann anhand dieser Variablen danach analysiert werden, ob die Norm dieser Handlung (laut Parsons wohnt jeder Handlung eine Norm inne) auf alle Personen anwendbar ist, oder nur für eine bestimmte Person oder Personengruppe gilt.
            \item
                Zuschreibung versus Leistung: Anhand dieser Variablen kann eine Handlung danach analysiert werden, ob sie gegenüber einer anderen Person oder Personengruppe aufgrund von Zuschreibung vorgenommen wurde, oder aufgrund von Leistungen und Verdiensten, die diese Person(en) erworben haben.
            \item
                Diffusität versus Spezifität: Alternative zwischen Handlungen, die auf die ganze Person (z.B. Familienvater: Rolle als Versorger, Erzieher, liebender Vater) und solchen, die auf spezielle Segmente, d. h. einzelne, klar definierte „Teile“ (Rollen) des Individuums bezogen sind (z.B. als Heizungsmonteur).
            \item
                Selbstorientierung versus Kollektivorientierung: Alternative zwischen Eigeninteressen (Eigennutz) und dem Bezug auf das Kollektivwohl (Gemeinnutz).
        \end{enumerate}
\end{itemize}

\subsection*{Normen}
\begin{itemize}
    \item
        Normen sind soziale (also überindividuelle) Festlegungen der Bedeutungen von Verhalten in einer Situation, sie legen unsere Erwartung an das Verhalten anderer fest
    \item
        Verbindlichkeitesgrade von Normen:
        \begin{center}
            \begin{tabular} { c | c | c | c }
                \hline
                Art der Erwartung & Positive Sanktion & Negative Sanktion & \\ \hline
                Muss & - & Gericht & Gesetz (Steuern zahlen)\\ \hline
                Soll & (Gratifikation) & Ausschluss & Sitte (Flirtregeln) \\ \hline
                Kann & Schätzung & (Antipathie) & Gewohnheit (Hilfsbereitschaft) \\ 
                \hline
            \end{tabular}
        \end{center}
    \item
        Abweichungen liegen nicht in der Sache, sondern entstehen relativ zu Erwartungen (z.B. Sucht, Krankheit)
    \item
        Sanktionen Verstärken eine Norm bei ihrer Erfüllung bzw. Verletzung
    \item
        Nicht jede Erwartungsenttäuschung führt zur Sanktion $\rightarrow$ korrektiver Austausch
    \item
        \textbf{Individualisierungschancen liegen nicht in der Konformität, sondern in der Abweichung}
    \item
        \textbf{Werte} rahmen und legitimieren Normen 
    \item     
        Werte: kulturell verbreitete Vorstellung des Wunschbaren, Erstrebenswerten, Wertvollen.
    \item
        Werte definieren ein allgemeines Ziel und geben eine Orientierung, sie legen kein konkretes Handeln fest
        \begin{center}
            \begin{tabular} { c | c | c }
                \hline
                Werte/Ziele akzeptiert & Normen/Mittel akzeptiert & Anpassungsmuster \\ \hline
                Ja & Ja & Konformismus \\ \hline
                Ja & Nein & Innovation \\ \hline
                Nein & Ja & Ritualismus \\ \hline
                Nein & Nein & Apatie und Rückzug \\ \hline
                Alternativ & Alternativ & Rebellion \\
                \hline
            \end{tabular}
        \end{center}
\end{itemize}
\subsection*{Culture of Poverty}
\begin{itemize}
    \item
        Behauptung: in Subkulturen werden Werte vermittelt, die nicht denen der dominanten Kultur entsprechen (sog. Armutskulturen)
    \item
        In Folge dessen orientieren sich Jugendliche aus den Subkulturen nicht an den gesellschaftlich zur Verfügung gestellten Mitteln (etwa Bildungsangebote), weil sie nicht dieselben Ziele anstreben
    \item
        Gegenbehauptung: Jugendliche favorisieren dieselben Werte, sozial akzeptierte Mittel um diese Ziele zu erreichen sind jedoch unvertraut $\rightarrow$ Kultur-Schock Erfahrungen, Reorientierung des Verhaltens auf kulturelle Milieus der Vertrautheit
\end{itemize}

\subsection*{Fazit}
\begin{itemize}
    \item
        Handeln: Folge von Abwägung und dadurch versteh- und gestaltbar
    \item
        Handeln hat Folgen und Nebenfolgen, Handeln orientiert sich an Normen
    \item
        Normen machen unser Handeln und das der anderen erwartbar, sie sind uns im Wissen präsent
    \item
        Norm und Abweichung sind wechselseitig aufeinander bezogene Begriffe
    \item
        Abweichendes Verhalten ist wichtige Quelle sozialen Wandels
    \item
        Anerkannte Ziele fördern auch abweichende Strategien ihres Erreichens, wenn die anerkannten Mittel vorenthalten werdern
    \item
        Werte wirken nicht nur integrativ, sondern auch ausschließend (Christliche Werte $\Rightarrow$ fordern Bekenntnis)
\end{itemize}

\newpage

\section*{Vorlesung 4: Integration, Kommunikation}
\subsection*{Normen und Interaktionskrisen}
\begin{itemize}
    \item
        Normen regulieren in Situationen Erwartungen an das eigene Verhalten und das Verhalten anderer. Viele solcher Normen sind uns selbstverständlich, wir befolgen sie als gehörten sie zu unserer Natur (gerade basale Integrationsnormen wie Grüßen)
    \item
        Durch Verletzung der Normen und Analyse der Reaktionen kann Wichtigkeit erkannt werden (je heftiger Reaktion, desto basaler die Norm) $\rightarrow$ Krisenexperiment
    \item
        Krisenexperiment: situatives unangemessenes Verhalten = bewusstes Stören der Basisregel der Interaktion $\rightarrow$ Akteure reagieren mit Versuchen der Normalisierung der Situation
    \item
        Wie regulieren Werte/Normen Interaktionen?
        \begin{itemize}
            \item
                Parson: Werte und Normen werden im Prozess der Sozialisation verinnerlich (einfacher Beleg: Schuldbewusstsein)
            \item
                Was folgt daraus für die Regulierung von Interaktionen? $/rightarrow$ Dass man kulturen Normen folgt, weil man sie verinnerlicht hat (stimmt nur halb)
            \item
                Interaktionistische Perspektive auf soziale Normen
            \item
                Beispiel Paar: Paarbeziehung folgt nicht immer den gleichen Normen, am Anfang der Beziehung noch keine stabilen Normen, Paar übersetzt Normen in ihre Paarbeziehung (Aushandlungsprozess)
        \end{itemize}
    \item
        Folgerung: Eine Norm gibt nicht an, wie sie angewendet wird, sie wird in Interaktionssituationen praktisch verfertigt (Kritik an Parson)
    \item
        Interaktionen sind Orte der Entstehung, der Anwendung und der Wandels der Normen, Werten und Institutionen
\end{itemize}

\subsection*{Interaktion}
\begin{itemize}
    \item
        Definition: mind. zwei Akteure, die ihr Verhalten wechselseitig aufeinander abstimmen - mittels der symbolischen Bedeutung ihres Tuns (z.B. Handwedeln bedeutet Grüßen)
    \item
        Deuten des Verhaltens anderer = Situationsdefinition (z.B. Dude sitzt da und sagt nichts $\rightarrow$ traurig oder vielleicht doch nur gelangweilt?)
    \item
        Handlungskoordination setzt eine gemeinsame Situationsdefinition voraus
    \item
        Interaktionen folgen in ihrem Verlauf nicht bzw. nicht nur Absichten, sondern der Wechselwirkung zwischen den Akteuren
    \item
        Symbolischer Interaktionismus:
        \begin{itemize}
            \item
                Wir handeln gegenüber Dingen und Anderen aufgrund von Bedeutungen derer
            \item
                Bedeutungen sind aus sozialen Interaktionen mit Mitmenschen abgeleitet, in denen sie durch das Handeln definiert werden
            \item
                Bedeutungen werden in einem interpretativen Prozess gehandhabt und verändert
            \item
                Bedeutung ist keine immanente Qualität, sondern ein Produkt der Interaktion im Umgang mit den Dingen und Menschen
            \item
                Interaktion: Austausch von Gesten und Symbolen zwischen mehreren Individuen, die miteinander und aufeinander bezogen handeln
        \end{itemize}
    \item
        Thomas-Theorem: \textit{if men define situations as real, they are real in their consequences} (Situationsdefinition, Orientierung des Handelns an der Definition, Reale Konsequenzen), Self fullfilling prophecy
\end{itemize}
\subsection*{Kommunikation}
\begin{itemize}
    \item
        Interaktion: Austausch von Gesten und Symbolen zwischen mehreren Individuen, die miteinander und aufeinander bezogen handeln
    \item
        Symbol: Zeichen, dessen Bedeutung geteilt wird (Kommunikative Welt = symbolische Welt)
    \item
        Kommunikation: menschliche Verhaltensabstimmung mittels symbolischer Mittel, die in soziale Praktiken eingebettet sind (Reichertz)\\
    \item
        Kommunikation ist aus drei Gründen nach Luhmann unwahrscheinlich:
        \begin{itemize}
            \item
                Jeder versteht die Welt auf der Grundlage seiner individuellen Biografie und deshalb anders als ein anderer (Unwahrscheinlichkeit des Verstehens)
            \item
                Dass eine Kommunikation mehr Personen erreicht, als in einer Situation anwesend sind, ist unwahrscheinlich (Unwahrscheinlichkeit des Erreichens Abwesender)
            \item
                Dass ein Adressat eine Mitteilung als Prämisse seines Verhaltens übernimmt, ist unwahrscheinlich (Unwahrscheinlichkeit der Annahme)
        \end{itemize}
    \item
        Luhmanns Antwort: Die Unwahrscheinlichkeit wird reduziert durch Kommunikationsmedien
    \item
        Kommunikation setzt Hör-, Sicht-, oder Spürbarkeit des kommunikativen Vorgangs voraus
    \item
        Kommunikationsmedien haben ein materiales Substrat (Verbreitungsmedien und Erfolgsmedien)
    \item
        Kommunikation = Einheit von Information, Mitteilung und Verstehen (Was gesagt wird, wie es gesagt (Kontext) wird und was ein Adressat daraus macht)
    \item
        Beispiel Sprache:
        \begin{itemize}
            \item
                Ermöglicht Verstehen
            \item
                Macht Annahme der Mitteilung unwahrscheinlich
            \item
                Ja/Nein-Kodierung der Sprache
            \item
                Öffnet Raum der Kontingenz
            \item
                Folgerung: Der Fortgang von Kommunikation liegt nie fest (lässt sich nicht intentional von einer Seite steuern), ist aber auch nicht beliebig
        \end{itemize}
    \item
        Interaktion ist an Kopräsenz gebunden, Kommunikation nicht
    \item
        Ergebnis: Kommunikationsmedien sind institutionalisierte (sozial verfestigte, stabile) Lösungen von Koordinationsproblemen
\subsection*{Fazit}
    \begin{itemize}
        \item
            Differnz handlungstheoretischer / interaktionistischer Perspektive auf soziale Phänomene
        \item
            Interaktion funktioniert auf der Grundlage geteilter Symbole (etwa: Sprache)
        \item
            Folgerungen: u.a. Thomas-Theorem
        \item
            Begriff der Interaktion verweist auf Kommunikation bzw. Kommunikationsmedien
        \item
            Kommunikation in Interaktionen basiert auf (geteilten) Symbolen
        \item
            Keine Interaktion ohne Kommunikation (umgekehrt schon)
    \end{itemize}
\end{itemize}

\subsection*{Vorlesung 5: Institution, Organisation, kollektives Handeln}
\begin{itemize}
    \item
        Handlungstheoretisch: Soziale Phänomene sind in den handelnden Akteuren zu verorten
    \item
        Interaktionistisch: Soziale Phänomene sind in den Wechselbeziehungen von Akteuren zu verorten
    \item
        Institutionalistisch: Soziale Phänomene sind unabhängig von den Akteuren und üben auf diese einen Zwang aus
    \item
        Beispiel Paar:
        \begin{itemize}
            \item
                aus der Perspektive von A und B (Normen der Paarbeziehung werde durch Verinnerlichung handlungsleitend): Man schaut auf die einzelnen Handelnden
            \item
                aus der Perspektive der Wechselwirkung (Normen der Paarbeziehung werden kommunikativ ausgehandelt): Man schaut auf die Interaktion
            \item
                aus der Perspektive der Institutionen (Normen der Paarbeziehung besehen unabhängig von Akteuren): Man schaut auf die Normen
        \end{itemize}
\end{itemize}
\subsubsection*{Institution}
\begin{itemize}
    \item
        Institutionen: Sinneinheit von habitualisierten (selbst ins Verhalten übernommenen) Formen des Handelns und der sozialen Interaktion, deren Sinn und Rechtfertigung der jeweiligen Kultur entstammen und deren dauerhafte Beachtung die Gesellschaft sichert\\
        Beispiel: Paarwerdung als Prozess der Institutionalisierung von sozialen Positionen in der Beziehung
    \item
        Institutionalsierung:
        \begin{itemize}
            \item
                Stabilisierung von Verhaltenserwartungen und Problemlösungen (Handlungskoordination)
            \item
                Einschränkung der Vielfalt von Handlungsmöglichkeiten
            \item
                Stabilisierung sozialer Ordnungen durch Ablösung der Institution von der Person
        \end{itemize}
    \item
        Folge: Das Soziale scheint uns fest wie dingliche Objekte, das Soziale scheint uns selbstverständlich
    \item
        Institutionen sind darauf angewiesen, dass sie im Handeln reproduziert werden
    \item
        Dimensionen von Institutionen:
        \begin{itemize}
            \item
                Ideel (Was)
            \item
                Personell (Wer)
            \item
                Regeln und Normen 
            \item
                Materieller Apparat
        \end{itemize}
\end{itemize}
\subsection*{Organisation}
\begin{itemize}
    \item
        Im Unterschied zu Institutionen absichtlich geschaffen
    \item
        Verfolgen Zwecke, definieren Mitgliedschaft und bilden Hirarchien aus
    \item
        soziale Phänomene, keine sachlich fassbaren Gegenstände wie Autos
    \item
        Integrieren ihre Mitglieder nur in spezifischen, von Personen unabhängigen Funktioinen, nicht als ganze Person
    \item
        Moderne Sozialordnungen werden auch als Organistationsgesellschaft bezeichnet
    \item
        Beispiele: Wirtschaftsbetriebe, Schulen, Hochschulen, Krankenhäuser, Parteien, Vereine, Kirchen, Gerichte, Staaten, ...
    \item
        Nach Parson: wichtigster Mechanismus für eine hochdifferenzierte Gesellschaft, um das System in Gang zu halten und Ziele zu verwirklichen, die die Möglichkeiten des einzelnen übersteigen
    \item
        Ziehen eine Grenze zwischen sich und ihrer Umwelt, bauen intern Komplexität auf um mit ihrer Umwelt umgehen zu können. Das tun sie durch funktionale Spezialisierung und Hirarchieaufbau
    \item
        Eigenlogik von Organisationen:
        \begin{itemize}
            \item
                Parkinsons Gesetz (erklärt das beständige Wachsten von Bürokratie)
                \begin{itemize}
                    \item
                        Angstellte wünschen, die Zahl ihrer Untergebenen, nicht die Zahl ihrer Konkurrenten zu vermehren
                    \item
                        Angstellte schaffen sich gegenseitig Arbeit
                \end{itemize}
            \item
                Ergebis am Beispiel staatlicher Verwaltung: hohe interne Rationalität, die von Außen betrachtet nur Kopfschütteln auslösen kann
        \end{itemize}
    \item
        Mitgliedschaft:
        \begin{itemize}
            \item
                Herstellung von Konformität (Anerkennung von Organisationsregeln bei Beitritt, Ausschluss bei Nichtanerkennung) $\rightarrow$ Entkopplung subjektiver und organisationaler Motivlagen
            \item
                Formalisierung der Erwartungen an die Mitglieder = notwendiger Mechanismus, der die Erwartungen der Organisation für Mitglieder sichtbar macht (aber nicht alles kann formalisiert werden)
            \item
                Indifferenzzonen (unklar ob Erwartung noch im Rahmen ist oder nicht)
            \item
                Motivation von Mitgliedern (Materielle Anreize, Zwang, Attraktive Tätigkeit, ...)
            \item
                Gegenwart: Verflüssigung von Organisationsgrenzen infolge uneindeutiger Mitgliedschaftsregeln (z.B. Leiharbeit)
        \end{itemize}
    \item
        Zwecke:
        \begin{itemize}
            \item
                Ermöglichen Organisationen ein vereinfachtes Bild ihrer Umwelt
            \item
                In der Regel Zwecke nicht so eindeutig
            \item
                Formulierte Zwecke stellen vor allem Akzeptanz her (z.B. nachhaltig wirtschaften)
    \item
        Hirarchien:
            \begin{itemize}
                \item
                    Zur Selbststeuerung
                \item
                    Akzeptanz der Hirarchie ist Mitgliedschaftsbedingung (ermöglicht Handlungskoordination und (unpopuläre) Entscheidungen)
                \item
                    Unterschiedliche Kompetenzen auf unterschiedlichen Ebenen (Sachkompetenz in der Regel unten)
                \item
                    Formelle und informelle Kommunikationswege
                \item
                    Gegenwart: Kritik und Wandel von Organisationshirarchien
            \end{itemize}
        \end{itemize}
\end{itemize}
\end{document}
