\documentclass{scrartcl}

\usepackage{hyperref}
\usepackage{amssymb}
\usepackage{mathtools}

\newcommand{\ffrac}[2]{\ensuremath{\frac{\displaystyle #1}{\displaystyle #2}}}

\begin{document}

\section[title]{Logical Agents\footnote{From \textit{Russel, Norvig - Artifical Intelligence A Modern Approach}, Chapter 7}}
\subsection{Knowledge-based Agents}
\begin{itemize}
    \item
        Central component: \textbf{knowledge base}, which consists of \textbf{sentences}, which are expressed in a \textbf{knowledge representation language}
    \item
        \textbf{Axiom}: sentece is taken as given witout being derived from other sentences (which is calles \textbf{inference}
    \item
        Two basic operations: \textbf{TELL} and \textbf{ASK} 
    \item
        Agent program steps:
    \begin{enumerate}
        \item
            TELL knowledge base what is perceived
        \item
            ASK knowledge base what action should be performed
        \item
            TELL knowledge base which action was chosen and execute the action 
    \end{enumerate}
\end{itemize}

\subsection{The Wumpus World}
... (Only motivation, why logical agents are lit af)

\subsection{Logic} 
\begin{itemize}
    \item 
        Sentences are expressed according to the \textbf{syntax} of the representation language, which specifies all the sentences that are well formed
    \item
        A logic must also define the \textbf{semantics} or meaning of setences (the truth of each sentence with respect to each possible world), every sentence must either be true or false in standard logic
    \item
        Possible world/\textbf{model}: "assignment of real numbers to the variables"
    \item
        If a sentence $\alpha$ is true in model $m$, we say that $m$ satisfies $\alpha$ or somtimes $m$ is a model of $\alpha$
    \item
        \textbf{Entailment}: a sentence follows logically from another sentence. Notation: $\alpha \vDash \beta$
    \item
        \textit{Formally}: $\alpha \vDash \beta$ if and only if, in every model in which $\alpha$ is true, $\beta$ is also true. Notation $\alpha \vDash \beta$ if and only if $M(\alpha) \subseteq M(\beta)$
    \item
        \textbf{Model checking}: enumerate all possible models to check that $\alpha$ is true in all models in which \textit{KB} is true, that is, that $M(KB) \subseteq M(\alpha)$
    \item
        "Set of all consequences of $KB$ is a haystack, $\alpha$ is a needle. Entailment is like the needle being in the haystack; inference is finding it"
    \item
        If an inference algorithm $i$ can derive $\alpha$ from $KB$, we write $KB \vdash_{i} \alpha$ ("$\alpha$ is derived from $KB$ by $i$")
    \item
        An inference algorithm that derives only entailed sentences is calles \textbf{sound} or \textbf{truthpreserving}
    \item
        An inference algorithm is \textbf{complete} if it can derive any sentence that is entailed
\end{itemize}

\subsection{Propositional Logic: A Very Simple Logic}
\begin{itemize}
    \item
        \textbf{Atomic Sentences} consist of a single \textbf{proposition symbol} (e.g. P, Q, R, ...)
    \item
        Two proposition symbols with fixed meaning: \textit{True} is the always-true proposition, \textit{False} the always-false proposition 
    \item
        \textbf{Logical connectives}:
    \begin{itemize}
        \item
            $\lnot$: negation 
        \item
            $\land$: conjunction
        \item
            $\lor$: disjunction
        \item
            $\Rightarrow$: implication (premise and conclusion, if-then). A $\Rightarrow$ B is true, unless A is true and B is false.
        \item
            $\Leftrightarrow$: if and only if. A $\Leftrightarrow$ B is true, if A and B are both true or both false. 
    \end{itemize} 
    \item
        In propositional logic, a model simply fixes the truth value for every proposition symbol, e.g. $m_{1} = \{P_{1,2} = false, P_{2,2} = true, ...\}$ 
    \item
        Propositional entailment is co-NP-complete (probably not easier than NP-complete), so every known inference algorithm for propositional logic has a worst-case complexity that is exponential in the size of the input
\end{itemize}  

\subsection{Propositional Theroem Proving}
\begin{itemize}
    \item
        If the number of models is large, but the length of the proof is short, then theorem proving can be more efficient than model checking 
    \item
        \textbf{Logical equivalence}: Two sentences $\alpha$ and $\beta$ are logically  equivalent if they are true in the same set of models. Notation: $\alpha \equiv \beta$

    \textbf{Standard logical equivalences}:
    \label{logEqui}
    \begin{itemize}
        \item
            $\alpha \Rightarrow \beta \equiv \lnot \beta \Rightarrow \lnot \alpha$
        \item
             $\alpha \Rightarrow \beta \equiv (\lnot \alpha \lor \beta)$
        \item
            $\alpha \Leftrightarrow \beta \equiv ((\alpha \Rightarrow \beta) \land (\beta \Rightarrow \alpha))$
        \item
            $\lnot (\alpha \land \beta) \equiv (\lnot \alpha \lor \lnot \beta)$ ($\alpha \lor \beta$ analog)
        \item
            $\alpha \land (\beta \lor \gamma) \equiv ((\alpha \land \beta) \lor (\alpha \land \gamma))$  ($\alpha \lor (\beta \land \gamma)$ analog)
    \end{itemize} 
    \item
        Other definition: $\alpha \equiv \beta$ if and only if $\alpha \vDash \beta$ and $\beta \vDash \alpha$
    \item
        \textbf{Validity}: A sentence is valid if it is true in \textit{all} models (also called \textbf{tautologies})
    \item
        Every valid sentence is logically equivalent to \textit{True}
    \item
        \textbf{Deduction theorem}: \textit{For any sentences $\alpha$ and $\beta$, $\alpha \vDash \beta$ if and only if the sentence $(\alpha \Rightarrow \beta)$ is true in every model}
    \item
        \textbf{Satisfiability}: A sentence is satisfiable if it is true in, or satisied by, some model
    \item
        \textit{$\alpha \vDash \beta$ if and only if the sentence $(\alpha \land \lnot \beta)$ is unsatisfiable)} ("proof by contradiction", assumption that $\beta$ to be false leads to a contradiction with known axioms $\alpha$)
\end{itemize}

\subsubsection{Inference and proofs}
\begin{itemize}
    \item
        \textbf{Modus Ponens}: $\ffrac{\alpha \Rightarrow \beta, \; \alpha}{\beta}$
    \item
        \textbf{And-Elimination}: $\ffrac{\alpha \land \beta}{\alpha}$
    \item
        All logical equivalences (\hyperref[logEqui]{Click}) can be used as inference rules, they are all \textit{sound}
    \item
        Finding a proof can be more efficient because the proof can ignore irrelevant propositions, no matter how many of them there are
    \item
        \textbf{Monotonicity}: the set of entailed sentences can only increase as information is added to the knowledge base
    \item
        if $KB \vDash \alpha$ then $KB \land \beta \vDash \alpha$
    \item
        The conclusion of the rule must follow \textit{regardless of what else is in the knowledge base}
\end{itemize}

\subsubsection{Proof by resolution}
\begin{itemize}
    \item
        Search algorithms with inadequate number of available inference rules are not \textit{complete} 
    \item
        \textbf{unit resolution} inference rule: $\ffrac{l_{1} \lor \dots \lor l_{k}, \;\;\; m}{l_{1} \lor \dots \lor l_{i-1} \lor l_{i+1} \lor \dots l_{k}}$
        where $l_{i}$ and $m$ are \textbf{complementary literals} (one is the negation of the other)
    \item
        Takes a \textbf{clause} - a disjunction of literals - and a literal and produces a new clause. Note that a single literal can be viewed as a disjunction of one literal, also known as a \textbf{unit clause}
    \item
        \textbf{full resolution} rule: $\ffrac{l_{1} \lor \dots \lor l_{k}, \;\;\; m_{1} \lor \dots \lor m_{n}}{l_{1} \lor \dots \lor l_{i-1} \lor l_{i+1} \lor \dots \lor l_{k} \lor m_{1} \lor \dots \lor m_{j-1} \lor m_{j+1} \lor \dots \lor m_{n}}$
        where $l_{i}$ and $m_{j}$ are complementary literals
    \item
        The resulting clause should contain only one copy of each literal. The removal of multiple copies of literals is called \textbf{factoring}
    \item
        \textit{Soundness of resolution}: When $l_{i}$ true, then $m_{j}$ false, and hence $m_{1} \lor \dots \lor m_{j-1} \lor m_{j+1} \lor \dots \lor m_{n}$ must be true.
        If $l_{i}$ false, then $l_{1} \lor \dots \lor l_{i-1} \lor l_{i+1} \lor \dots \lor l_{k}$ must be true. Now, $l_{i}$ is either true or false, so one or other of these conclusions holds - exactly as the resolution rule states.
    \item
        \textit{A resolution-based theroem prover can, for any sentences $\alpha$ and $\beta$ in propositional logic, decide whether $\alpha \vDash \beta$}.
\end{itemize}

\subsubsection{Conjunctive normal form}
\begin{itemize}
    \item
        Resolution rule only applies to clauses (disjunctions of literals)
    \item
        \textit{But:} Every sentence of propositional logic is equivalent to a conjunction of clauses (\textbf{conjunctive normal form, CNF})
\end{itemize}
Converting a sentence (e.g. $B_{1,1} \Leftrightarrow (P_{1,2} \lor P_{2,1})$) into \textbf{CNF}:
\begin{enumerate}
    \item
        Eliminate $\Leftrightarrow$, replacing $\alpha \Leftrightarrow \beta$ with $(\alpha \Rightarrow \beta) \land (\beta \Rightarrow \alpha)$

        $(B_{1,1} \Rightarrow (P_{1,2} \lor P_{2,1})) \land ((P_{1,2} \lor P_{2,1}) \Rightarrow B_{1,1})$
    \item
        Eliminate $\Rightarrow$, replacing $\alpha \Rightarrow \beta$ with $\lnot \alpha \lor \beta$

        $(\lnot B_{1,1} \lor P_{1,2} \lor P_{2,1}) \land (\lnot (P_{1,2} \lor P_{2,1}) \lor B_{1,1})$
    \item
        CNF required $\lnot$ to appear only in literals, so we "move $\lnot$ inwards" by repeated application of the following equivalences:

        $\lnot(\lnot \alpha) \equiv \alpha$

        $\lnot(\alpha \land \beta) \equiv (\lnot \alpha \lor \lnot \beta)$

        $\lnot(\alpha \lor \beta) \equiv (\lnot \alpha \land \lnot \beta)$

        In the example:
        $(\lnot B_{1,1} \lor P_{1,2} \lor P_{2,1}) \land ((\lnot P_{1,2} \land \lnot P_{2,1}) \lor B_{1,1})$

    \item
        Eliminate nested $\lor$ and $\land$ operators by applyingdistributivity law

        $(\lnot B_{1,1} \lor P_{1,2} \lor P_{2,1}) \land (\lnot P_{1,2} \lor B_{1,1}) \land (\lnot P_{2,1} \lor B_{1,1,})$
\end{enumerate}

\subsubsection{A resolution algorithm}
\begin{itemize}
    \item
        To show that $KB \vDash \alpha$, we show that $(KB \land \lnot \alpha)$ is unsatisfiable
\end{itemize}
\textit{PL-Resolution Algorithm}:
\begin{enumerate}
    \item
        Convert $(KB \lor \lnot \alpha)$ into CNF
    \item
        Apply resolution rule to the resulting clauses. Each pair that contains complementary literals is resolved to produce a new clause, which is added to the set if not already present
    \item
        The process continues until one of two things happens:

        1) there are no new clauses that can be added, so $KB$ does not entail $\alpha$
        
        2) two clauses resolve to yield the empty clause, so $KB$ entails $\alpha$
\end{enumerate}
\begin{itemize}
    \item
        The empty clause - a disjunction of no disjuncts - is equivalent to \textit{False}
\end{itemize}

\textit{Completeness:}
\begin{itemize}
    \item
        \textbf{Resolution closure }\textit{RC(S)} of a set of clauses $S$: set of all clauses derivable by repeated application of the resolution rule to clauses in $S$ or their derivatived
    \item
        \textbf{Ground resolution theorem:} If a set of clauses is unsatisfiable, then the resolution closure of those clauses conteins the empty clause.
\end{itemize}

\subsubsection{Horn clauses and definite clauses}
\begin{itemize}
    \item
        Some real-world knowledge bases satisfy certain restrictions on the form of sentences they contain, which enables them to use a more restricted and efficient inference algorithm
    \item
        \textbf{definite clause:} disjunction of literals of which exactly one is positive, e.g. $(\lnot L_{1,1} \lor \lnot Breeze \lor B_{1,1})$
    \item
        \textbf{Horn clause:} disjunction of literals of which at most one is positive (so all definite clauses are Horn clauses, as are clauses with no positive literals (so called \textbf{goal clauses}))
    \item
        Horn clauses are closed under resolution: if you resolve two Horn clauses, you get back a Horn clause
\end{itemize}
Knowledge bases only containing definite clauses are interesting for three reasons:
\begin{enumerate}
    \item
        Definite clauses can be written as implications, whose preimse is a conjunction of positive literals and whose conclusion is a single positive literal. A sentence consisting of a single positive literal, such as $L_{1,1}$ is called a fact, it too can be written in implication form as $True \Rightarrow fact$
    \item
        Inference with Horn clauses can be done through the \textbf{forward-chaining} and \textbf{backward-chaining} algorithms (basis for \textbf{logic programming}
    \item
        Deciding entailment with Horn clauses can be done in time that is linear in the size of the knowledge base
\end{enumerate}

\subsubsection{Forward and backward chaining}
\begin{itemize}
    \item
        Forward chaining: begins from known facts (positive literals) in the knowledge base. If all the premises of an implication are known, then its conclusion is added to the set of known facts. This process continues until the query $q$ (a single propositional symbol) is added or until no further inferences can be made
    \begin{itemize}
        \item
            \textbf{Sound:} Every inference is essentially an application of Modus ponens
        \item
            \textbf{Complete:} Every entailed atomic sentence will be derived
    \end{itemize}
    
    \item
        Forward chaining is an example of the general concept of \textbf{data-driven} reasoning - that is, reasoning in which the focus of attention start with the known data.
    \item
        Backward chaining: works backwards from the query. If all the premesis of one of the implications whose conclusion is $q$ can be proven true, then $q$ is true.
    \item
        \textbf{goal-directed reasoning}. It is useful for answering specific questins such as "What shall I do now?" and "Where are my keys?".
    \item
        Both run in linear (often less) time
\end{itemize}

\subsection{Effective Propositional Model Checking}
    TODO

\subsection{Agents Based on Propositional Logic}
    TODO

\newpage

\section{First-Order Logic}
\subsection{Representation Revisited}
    TODO

\subsection{Syntax and Semantics of First-Order Logic}
\subsubsection{Models for first-order logic}
\begin{itemize}
    \item
        Models for first-order logic have objects
    \item
        \textbf{Domain} of a model: Set of objects or \textbf{domain elements} it contains, required to be non-empty (every possible world must contain at least one object)
    \item
        Relation: set of tuples of objects that are related
\end{itemize}

\subsubsection{Symbols and interpretations}
\begin{itemize}
    \item
        Basic syntactic elements: symbols that stand for objects, relations and functions
    \item
        Three kinds: \textbf{constant} (stand for objects), \textbf{predicate} (stand for relations) and \textbf{function symbols} (stand for functions)
    \item
        Predicate and function symbols come with an \textbf{arity} (Stelligkeit), that fixes the number of arguments
    \item
        Each model includes an \textbf{interpretation} that specifies exactly which objects, relations and functions are referred to by the constant, predicate, and function symbols
\end{itemize}

\subsubsection{Terms}
\begin{itemize}
    \item
        \textbf{Term}: Logical expression that refers to an object
    \item
        It's not always convenient to have a distinct symbol to name every object
    \item
        $f(t_1, \dots, t_n)$. Function symbol $f$ refers to some function in the model. The argument terms refer to objects in the domain. The term as a whole refers to the object that is the value of the function applied.
\end{itemize}

\subsubsection{Atomic sentences}
\begin{itemize}
    \item
        \textbf{Atomic sentence}: state facts
    \item
        Is formed from a predicate symbol optionally followed by a parenthesized list of terms
    \item
        \textit{An atomic sentence is true in a given model if the relation referred to by the predicate symbol holds among the objects referred to by the arguments}
\end{itemize}

\subsubsection{Complex sentences}
\begin{itemize}
    \item
        We can use \textbf{logical connectives} to consturct more complex sentences
\end{itemize}

\subsubsection{Quantifiers}
\begin{itemize}
    \item
        \textbf{Quantifiers} let us express properties of entire collections of objects 
\end{itemize}
\textbf{Universal quantification ($\forall$)}
\begin{itemize}
    \item
        $\forall x P$, where $P$ is any logical expressions, says that $P$ is true for every object $x$
    \item
        Usually use $\Rightarrow$ in $P$
\end{itemize}
\textbf{Existential quantification ($\exists$)}
\begin{itemize}
    \item
        $\exists x P$ says that $P$ is true for at least on object $x$
    \item
        Usually use $\land$ in $P$
    \item
        Implications lead to very weak statements (because if premise of conclusion is false, the existential statement is fullfilled)
\end{itemize}
\textbf{Connections between $\forall$ and $\exists$}
$$\forall x \lnot P \equiv \lnot \exists x P$$
$$\lnot \forall x P \equiv \exists x \lnot P$$
$$\forall x P \equiv \lnot \exists x \lnot P$$
$$\exists x P \equiv \lnot \forall x \lnot P$$

\subsubsection{Equality}
\begin{itemize}
    \item
        We can use \textbf{equality symbols} to signify that two terms refer to the same object
\end{itemize}

\subsubsection{An alternative semantic?}
\begin{itemize}
    \item
        Very popular proposal in database systems (\textbf{database semantics})
    \item
        \textbf{unique-name assumption:} Every constant symbol refers to a distinct object
    \item
        \textbf{closed-world assumption:} Atomic sentences not known to be true are in fact false
    \item
        \textbf{domain closure:} Each model contains no more domain elements than those named by the constant symbols
\end{itemize}

\subsection{Using First-Order Logic}
\subsubsection{Assertions and queries in first-order logic}
    TODO
\end{document}
